\documentclass[a4paper, 12pt] {report}

\usepackage{setspace}
\usepackage{fullpage}
\usepackage{listings}
\usepackage{graphicx}


\author{Caleb Everett}
\date{\today}
\title{Vector Adder}

\begin{document}
\maketitle

\section*{Introduction}{}
This program is modeled after most interactive shells, or command prompts. The
program prompts the user for input, which is interpreted as a command to execute, or
a vector to input.  After the vector has been added, or the command completed, the
program will prompt the user for input again, and the process will be repeated until
the `exit' command is entered.  When the user enters a vector it is stored in an
array, and when prompted will print the total of all vectors in the array. The
program uses a set of known units to automatically convert all vectors to the same
unit when adding them together.  There is a command to set the default output unit as
well.

\section*{Manual}
\subsection*{Commands}
\begin{center}
  \begin{tabular}{| l | l | p{12cm} |}
    \hline
      Command & Arguments & Description \\
    \hline
         exit &           & Exits the program \\ 
    \hline
        clear &           & Clears the terminal, it's possible this will not work.
                            Tested and working on linux and windows. \\
    \hline
         list &           & Lists all entered vectors, their indexes and the sum.
                            Alias 'ls'. \\ 
    \hline
       remove & IDX       & Removes the vector at the supplied IDX, if IDX is 
                             'all', all vectors are removed. When a vector 
                             is removed its value is removed from the total as well.
                             Alias 'rm'. \\
    \hline
          sum &           & Prints the sum of all entered vectors. \\
    \hline
        units & UNIT      & Sets the units the sum will be output in. No unit is 
                             supplied units are reset to default.  Default behavior 
                             for units it to output in the units of the first vector. \\
    \hline
         help &           & Prints a help message listing these commands. \\
    \hline
  \end{tabular}
\end{center}

\pagebreak
\subsection*{Vectors}
Input in the following forms will be interpreted as a vector. Including commas or
parentheses is optional, but each token must be separated by at least one of the
following characters: space, comma, parentheses, or tab. When a vector is
successfully added it is printed in cartesian form, with it's index in the array.
The following units are understood: kN, N, lb. Capitalization matters.
\begin{center}
  \begin{tabular}{| c | l |}
    \hline
      cartesian & Vector in cartesian format \\
                & F unit F unit F unit \\
                & 5 N 2 N 4 N \\
                & 2 lb, 3 N, 3 N \\
    \hline
      point to point & Vector defined by a force and a line \\
                     & F unit (x,y,z) (x,y,z) \\
                     & 5 N (0,0,0) (1,2,3) \\
    \hline
      coordinate angle & Force and at least two coordinate angles, \\
                       &  specify which angles are given with a, b, y\\
                       & F unit deg a deg b deg y \\
                       & 5 N 90 a 90 b 0 y \\
                       & 5 N 45 a 45 y \\
    \hline
  \end{tabular}
\end{center}

\subsection*{Example Problems}
\lstset{language=bash, showstringspaces=false, basicstyle=\scriptsize, 
  columns=fixed, extendedchars=true, 
  frame=single, breaklines=true}

  Lines marked by `-$>$' indicate user input and lines without indicate output from the 
  program.  To see the sum of all inputted vectors enter the command `sum', which will 
  print the sum in cartesian and coordinate forms.
  \subsubsection*{2.90}
    input:
    \begin{lstlisting}
    -> 600 N (0,2,4) (0,0,0)
    Added vector 0: {0 Ni, -268 Nj, -537 Nk}
    -> 500 N (0,2,4) (4,8,0)
    Added vector 1: {243 Ni, 364 Nj, -243 Nk}
    \end{lstlisting}
    answer:
    \begin{lstlisting}
    -> sum
    sum: {243 Ni, 95.5 Nj, -779 Nk}
         822 N, a = 72.8, b = 83.3, y = 162
    \end{lstlisting}

  \subsubsection*{2.96}
    input:
    \begin{lstlisting}
    -> 400 N (0,0,24) (15,20,0)
    Added vector 0: {173 Ni, 231 Nj, -277 Nk}
    -> 800 N (0,0,24) (-6,4,0)
    Added vector 1: {-192 Ni, 128 Nj, -766 Nk}
    -> 600 N (0,0,24) (16,-18,0)
    Added vector 2: {282 Ni, -318 Nj, -424 Nk}
    \end{lstlisting}
    answer:
    \begin{lstlisting}
    sum: {264 Ni, 40.9 Nj, -1.47e+03 Nk}
         1.49e+03 N, a = 79.8, b = 88.4, y = 170
    \end{lstlisting}

  \subsubsection*{2.102}
    input:
    \begin{lstlisting}
    -> 450 N (0,0,7) (-1.5,2.59,0)
    Added vector 0: {-88.7 Ni, 153 Nj, -414 Nk}
    -> 450 N (0,0,7) (-1.5,-2.59,0)
    Added vector 1: {-88.7 Ni, -153 Nj, -414 Nk}
    -> 450 N (0,0,7) (3,0,0)
    Added vector 2: {177 Ni, 0 Nj, -414 Nk}
    \end{lstlisting}
    answer:
    \begin{lstlisting}
    sum: {-0.0641 Ni, 0 Nj, -1.24e+03 Nk}
         1.24e+03 N, a = 90, b = 90, y = 180
    \end{lstlisting}

\subsection*{Compiling}
This project includes a gnu make file, if your system has make you can compile using

\lstset{language=bash, showstringspaces=false, basicstyle=\scriptsize, 
  columns=fixed, extendedchars=true, 
  frame=single, breaklines=true}
\begin{lstlisting}
  $ make
\end{lstlisting}

If you do not have make use one of the following commands
linux/unix/mac:
\begin{lstlisting}
  $ g++ *.cpp -o vector.exe -lcurses
\end{lstlisting}

Windows:
\begin{lstlisting}
  $ g++ *.cpp -o vector.exe
\end{lstlisting}

If you do not have g++ on your system use what ever compiler you use, just make sure to 
include `curses' if on a posix system

\pagebreak

\section*{Flow Chart}
\includegraphics[width=1\textwidth, keepaspectratio=true]{HageComp1.pdf}
\pagebreak

\section*{Source Code}
\lstset{numbers=left,
        language=C++, 
        showstringspaces=false, 
        basicstyle=\scriptsize,
        aboveskip={1.5\baselineskip},
        columns=fixed,
        extendedchars=true,
        frame=single,
        breaklines=true}
\lstinputlisting{main.cpp}
\lstinputlisting{force_vector.h}
\lstinputlisting{force_vector.cpp}
\lstinputlisting{unit_num.h}
\lstinputlisting{unit_num.cpp}
\lstinputlisting{convert.h}
\lstinputlisting{clear_term.h}
\lstinputlisting{clear_term.cpp}

\end{document}
